%!TEX root =  flsa.tex
\textbf{Conclusion:} 
Step-size choice is critical in design of LSA algorithms, and especially in the case of TD algorithms, step-sizes are often treated as hyper-parameters that need to be tuned in a problem instance specific manner. To avoid this tuning, it is desirable to choose a single \emph{universal} step-size rule that works for all the instances in a problem class. This paper investigated the promise of an approach called CALSA (constant step-size averaged - LSA), an idea that goes back to \citet{ruppert} and \citet{polyak-jusidky}. For a given problem class, we asked $i)$ whether a \emph{universal} constant step-size can be chosen and $ii)$ whether a \emph{uniform} rate of convergence for the MSE can be achieved, across the class. We showed that answers to these questions in general is \emph{no}. However, we showed (under our assumptions) that any CALSA achieves convergence rate of $O(\frac{C_P}{t})$ for MSE, where the constant $C_P>0$ is problem instance dependent. We then showed that for some interesting problem classes in RL, TD algorithms with a problem independent universal constant step-size and iterate averaging, achieve a uniform asymptotic rate of $O(\frac{1}{t})$. 