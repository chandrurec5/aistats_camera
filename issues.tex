\documentclass{article}
\usepackage[a4paper]{geometry}
%!TEX root =  flsa.tex
%Packages
\usepackage{amsmath}
\usepackage{amsthm,amssymb}
\usepackage{comment}
\usepackage{amsfonts}
\usepackage{graphicx}
\usepackage{multirow}% http://ctan.org/pkg/multirow
%\usepackage{ntheorem}
%\usepackage[textsize=tiny]{todonotes}
\usepackage{tikz}
\usepackage{pgfplots}
\usepackage{enumitem}
\usepackage{algorithm}
\usepackage{algorithmic}
\usepackage{pifont}
\usepackage{makecell}
\usepackage{slashbox}
%!TEX encoding = UTF-8 Unicode\usepackage[numbers, sort, comma, square]{natbib}
%New Commands
\DeclareMathOperator{\supp}{supp}
\newcommand{\ip}[1]{\langle#1\rangle}
\newcommand{\norm}[1]{\left\| #1\right\|}
\newcommand{\normsm}[1]{\| #1\|}
\newcommand{\eqdef}{\stackrel{\cdot}{=}}
\setlength{\marginparwidth}{13ex}
\newcommand{\todoc}[2][]{\todo[size=\scriptsize,color=blue!20!white,#1]{Csaba: #2}}
\newcommand{\todoch}[2][]{\todo[size=\scriptsize,color=red!20!white,#1]{Chandru: #2}}
\usepackage[disable]{todonotes}
\usepackage{todonotes}
\usepackage{placeins}
\usepackage{xspace}

\newcommand{\vh}{\hat{V}_\pi}
\newcommand{\snorm}[1]{\left\|#1\right\|}
\newcommand{\dcd}{d \times d}
\newcommand{\err}{\emph{err}}
\newcommand{\B}{\mathcal{B}}
\newcommand{\V}{\mathcal{V}}
\newcommand{\nn}{\nonumber}
\newcommand{\cond}[1]{\kappa(#1)}
\newcommand{\md}[1]{\left|#1\right|}
\newcommand{\rhod}[1]{\rho_d(\alpha,#1)}
\newcommand{\rhos}[1]{\rho_s(\alpha,#1)}
\newcommand{\alphaps}{\alpha_s(P)}
\newcommand{\alphapd}{\alpha_d(P)}
\newcommand{\ra}{\rightarrow}
\newcommand{\zero}{\mathbf{0}}
\renewcommand{\P}{\mathcal{P}}
\newcommand{\pc}{p_{\mathcal{c}}}
\newcommand{\E}{\mathbf{E}}
\newcommand{\F}{\mathcal{F}}
\newcommand{\R}{\mathbb{R}}
\newcommand{\T}{\mathcal{T}}
\newcommand{\EE}[1]{\mathbf{E}\left[#1\right]}
\newcommand{\EEP}[1]{\mathbf{E}_P\left[#1\right]}
\newcommand{\eep}[2]{\mathbf{E}_{#2}\left[#1\right]}
\newcommand{\gln}{\mathrm{GL}(d)}
\newcommand{\gld}{\mathrm{GL}(d)}
\newcommand{\ncn}{n\times n}
\newcommand{\I}{\mathcal{I}} % something better?
\newcommand{\C}{\mathbb{C}}
\newcommand{\re}[1]{\emph{re}(#1)}
\newcommand{\im}[1]{\emph{im}(#1)}
\newcommand{\op}{\oplus}
\newcommand{\tL}{\tilde{\Lambda}}
\newcommand{\tJ}{\tilde{J}}
\newcommand{\Lt}{\Lambda_t}
\newcommand{\ts}{\theta_*}
\newcommand{\eb}{\bar{e}}
\newcommand{\tb}{\bar{\theta}}
\newcommand{\zb}{\bar{z}}
\newcommand{\zh}{\hat{z}}
\newcommand{\eh}{\hat{e}}
%\newcommand{\rhoD}{\rho_D(\alpha,U,A)}
%\newcommand{\rhoR}{\rho_R(\alpha,U,A)}
\newcommand{\SD}{S^D_{\alpha,U}}
\newcommand{\SR}{S^R_{\alpha,U}}
\newcommand{\bu}{\beta(\alpha,U,P)}
\newcommand{\thh}{\hat{\theta}}
\newcommand{\gh}{\hat{\gamma}}
\newcommand{\iid}{\emph{i.i.d.}\xspace}

\newcommand{\M}{\mathcal{M}}
\renewcommand{\P}{\mathcal{P}}
\renewcommand{\S}{\mathcal{S}}
\renewcommand{\R}{\mathcal{R}}
\newcommand{\A}{\mathcal{A}}

%Theorem Definition
\theoremstyle{definition}
\newtheorem{theorem}{Theorem}
\newtheorem{example}{Example}
\newtheorem{remark}{Remark}
\newtheorem{domain}{Domain}
\newtheorem{condition}{Condition}
\newtheorem{definition}{Definition}
\newtheorem{corollary}{Corollary}
\newtheorem{lemma}{Lemma}
\newtheorem{proposition}{Proposition}
%\theoremstyle{assumption}
\newtheorem{assumption}{Assumption}
%%%%%%%%%%%%%%%%%%%%%%%%%%%%%%%%%%%%%%%%%%%%%%%%
% Theorem environments and cleveref
%%%%%%%%%%%%%%%%%%%%%%%%%%%%%%%%%%%%%%%%%%%%%%%%
%\newtheorem{assumption}{Assumption}
\usepackage[capitalize]{cleveref}
\crefname{assumption}{Assumption}{Assumption}
\usepackage{thmtools}
\usepackage{thm-restate}

\let\lemma\relax
\declaretheorem[name=Lemma,refname={Lemma,Lemmas},Refname={Lemma,Lemmas},sibling=theorem]{lemma}

\let\corollary\relax\declaretheorem[name=Corollary,refname={Corollary,Corollaries},Refname={Corollary,Corollaries},sibling=theorem]{corollary}

\declaretheorem[name=Assumption,refname={Assumption,Assumptions},Refname={Assumption,Assumptions}]{assumption}

\let\proposition\relax\declaretheorem[name=Proposition,refname={Proposition,Propositions},Refname={Proposition,Propositions},sibling=theorem]{proposition}

\let\definition\relax
\declaretheorem[name=Definition,refname={Definition,Definitions},Refname={Definition,Definitions}]{definition}

% \crefname{question}{question}{questions}
\Crefname{question}{Question}{Questions}
\creflabelformat{question}{(#2#1#3)}

% \crefname{problem}{problem}{problems}
\Crefname{problem}{Problem}{Problems}
\creflabelformat{problem}{(#2#1#3)}

% \crefname{equation}{equation}{equations}
\Crefname{equation}{Equation}{Equations}
\creflabelformat{equation}{(#2#1#3)}

%\crefname{iCondition}{\ccap{c}ondition}{\ccap{c}onditions}
\crefname{iCondition}{Condition}{Conditions}
\creflabelformat{iCondition}{(#2#1#3)}
\crefrangelabelformat{iCondition}{(#3#1#4) to (#5#2#6)}

\Crefname{item}{}{}
\creflabelformat{item}{(#2#1#3)}
\crefrangelabelformat{item}{(#3#1#4) to (#5#2#6)}

%%%%%%%%%%%%%%%%%%%%%%%%%%%%%%%%%%%%%%%%


\newcommand{\Prob}[1]{\mathbb{P}\left(#1\right)}
% if you need to pass options to natbib, use, e.g.:
 \PassOptionsToPackage{compress}{natbib}
 \usepackage{natbib}
% before loading nips_2017
%
% to avoid loading the natbib package, add option nonatbib:
% \usepackage[nonatbib]{nips_2017}

%\usepackage{nips_2017}

% to compile a camera-ready version, add the [final] option, e.g.:
% \usepackage[final]{nips_2017}

\usepackage[utf8]{inputenc} % allow utf-8 input
\usepackage[T1]{fontenc}    % use 8-bit T1 fonts
\usepackage{hyperref}       % hyperlinks
\usepackage{url}            % simple URL typesetting
\usepackage{booktabs}       % professional-quality tables
\usepackage{amsfonts}       % blackboard math symbols
\usepackage{nicefrac}       % compact symbols for 1/2, etc.
\usepackage{microtype}      % microtypography
%\usepackage{varioref}
%\usepackage{thmtools}
\usepackage{aliascnt}
\newcommand{\tcite}[1]{\citeauthor{#1}~\citeyear{#1}}
\usepackage{titlecaps}
\usepackage[capitalize]{cleveref}
\usepackage{amsmath,amssymb,graphicx}
\crefname{assumption}{Assumption}{Assumption}

% BIBLIOGRAHY -----------------------------------------------------------------------------
% Option 1: bibtex natbib
%\usepackage[numbers]{natbib}
%\usepackage[colorlinks=true,linkcolor=blue,citecolor=purple,pagebackref=true]{hyperref}
% Option 2: biblatex
%\usepackage[backend=bibtex,natbib=true,backref=false]{biblatex}
%\usepackage[backend=bibtex,style=ieee,natbib=true,backref=false]{biblatex}
%\addbibresource{ref.bib}
% Adding back references to bib entries:
%\DefineBibliographyStrings{english}{%
%backrefpage = {page},% originally "cited on page"
%backrefpages = {pages},% originally "cited on pages"
%}
% Title, authors, etc. -------------------------------------------------------------------------



\title{Finite Time Bounds for Temporal Difference Learning with Function Approximation: Problems with some ``state-of-the-art'' results}

% The \author macro works with any number of authors. There are two
% commands used to separate the names and addresses of multiple
% authors: \And and \AND.
%
% Using \And between authors leaves it to LaTeX to determine where to
% break the lines. Using \AND forces a line break at that point. So,
% if LaTeX puts 3 of 4 authors names on the first line, and the last
% on the second line, try using \AND instead of \And before the third
% author name.

\author{
Chandrashekar Lakshmi Narayanan
\and
Csaba Szepesv\'ari 
%  David S.~Hippocampus\thanks{Use footnote for providing further
%    information about author (webpage, alternative
%    address)---\emph{not} for acknowledging funding agencies.} \\
%  Department of Computer Science\\
%  Cranberry-Lemon University\\
%  Pittsburgh, PA 15213 \\
%  \texttt{hippo@cs.cranberry-lemon.edu} \\
  %% examples of more authors
  %% \And
  %% Coauthor \\
  %% Affiliation \\
  %% Address \\
  %% \texttt{email} \\
  %% \AND
  %% Coauthor \\
  %% Affiliation \\
  %% Address \\
  %% \texttt{email} \\
  %% \And
  %% Coauthor \\
  %% Affiliation \\
  %% Address \\
  %% \texttt{email} \\
  %% \And
  %% Coauthor \\
  %% Affiliation \\
  %% Address \\
  %% \texttt{email} \\
}
\date{}

\begin{document}
% \nipsfinalcopy is no longer used

\maketitle
\begin{abstract}
In all branches of mathematics, including learning theory,
results build on previous results. Thus, it is important to keep the literature free of erroneous claims.
This short report lists some problems with the proofs and claims in the recent papers by \cite{flstd,lstdicml}, whose longer version containing the proofs are available on arxiv \citep{flstda,lstdicmla}. 
In particular, it follows that the results in these papers, if true, would need completely new proofs, and thus should not be used in the form stated by the authors.
\end{abstract}
\section{Introduction}
This short report lists some problems with the proofs of the claims in two recent papers  by \cite{flstd,lstdicml}.
As the problems seem serious, our conclusion is that to obtain results similar to those claimed in these two works, the assumptions of the stated claims need to be considerably strengthened and the form of the results will also need to be adjusted in significant ways.

The report is not self contained and its scope is limited to an audience who are interested in RL and TD(0) and specifically the above-mentioned papers.
Furthermore, since \cite{flstd,lstdicml} omitted the proofs, we will instead discuss their longer version in what follows, which have identical statements but include the proofs. The longer version of the paper by \cite{flstd} is \citep{flstda}, while the longer version
of \cite{lstdicml} is \citep{lstdicmla}.
We will  borrow the notation directly from these works.
%While the arguments presented below are about technical issues in the analysis, we don't provide any proofs as such and hence the presentation turns out to be a commentary of the prior results.\par
%The report is organized as follows. In \Cref{sec:flstd} we discuss the issues with the approach in \cite{flstda} and in \Cref{sec:lstdicml} the roadblocks as seen in \cite{lstdicmla} are presented. Both sections have other subsections which deal with the various separate difficulties.

\section{Expected Error Bound}

\subsection{Bugs in the paper by \cite{flstda}}
One of the main results of \citet{flstda} is Theorem 1, which states a bound on the expected error.
The proof of this theorem can be found in section A.2, starting on page 16.
The proof up to Eq. (26) is correct (in the definition of $M_{n+1}$, $F_n$ should be $F$).
However, after this display, we are told that by (A4), which ensures that $\frac1T \sum_{i=1}^T \phi(s_i) \phi_t(s_i)^\top \succ \mu I$ for some $\mu>0$, $\bar A_n = \frac1n \sum_{t=1}^n \phi_t (\phi_t - \beta \phi_t')^\top$ is such that $\bar A_n - (1-\beta) \mu I $ is positive definite. 
Here, $n>0$ is an arbitrary index and for $t\ge 1$ we use the abbreviation $\phi_t = \phi(s_{i_t})$ and $\phi_t' = \phi(s_{i_t}')$, where $i_t\in \{1,\dots,T\}$ is a random index.
\begin{quote}
In general, (A4) does \emph{not} imply that  $\bar A_n - (1-\beta) \mu I $ is positive definite.
\end{quote}
Take for example $n=1$. We would need that $\phi_1 (\phi_1 - \beta \phi_1')^\top - (1-\beta) \mu I$ is positive definite.
It is easy to construct examples where this is not true: Nothing prevents, for example, $\phi_1 = \beta \phi_1'$, in which case $\bar A_1 - (1-\beta)\mu I = -(1-\beta)\mu I$ is \emph{negative} definite.
(Note that the matrices involved are \emph{not} symmetric. Unfortunately, none of the two papers defines what is meant by positive definite in this case. We assume that the definition used is that a square, real-valued matrix $A$ is called positive definite if $x^\top A x\ge 0$ for any $x$ real-valued vector of appropriate dimension.)
In fact, we don't see why the claimed relationship would hold even when 
$\bar A_n$ is replaced by $\hat A_T \doteq  \frac1T \sum_{i=1}^T \phi( s_i) (\phi(s_i) - \beta \phi(s_i'))^\top$,
and we in fact suspect that this claim is false in full generality. But at minimum, a proof would be required and the whole subsequent argument will need to be changed.
\subsection{Bugs in the paper by  \cite{lstdicmla}}
In page 14 the expression below ``$A$ is a possibly random matrix...'' is not justified (personal communication with one of the authors confirmed this).  In particular, the claim here is that if $A$ is a random matrix with $\norm{A}_2\le C$ with $C$ a deterministic constant then for any $\theta$ deterministic vector,
\begin{align*}
\EE{ \theta^\top A^\top \EE{ \epsilon_n | \mathcal{F}_n } A \theta \,|\, s_0 } \le C^2 \theta^\top \EE{ \epsilon_n|s_0 } \theta\,.
\end{align*}
Recall that here $\epsilon_n$ is a matrix of appropriate dimensions, the ``mixing-error term''
and is actually $\mathcal{F}_n$ measurable ($\epsilon_n = \EE{ a_n | \mathcal{F}_n } - \mathbb{E}_{\Psi}[a_n]$). 
When the Markov chain is started from its stationary state 
(which is not ruled out by the conditions of the theorem under question), $\EE{ \epsilon_n|s_0 } = 0$.
If the above inequality was true, we would get 
\begin{align*}
\EE{ A^\top \epsilon_n  A  \,|\, s_0 } = 0\,.
\end{align*}
However, letting $B = \EE{\epsilon_n|\mathcal{F}_n}$ and, for example,
$A = C B/\norm{ B }_2$, we have 
\begin{align*}
A^\top \EE{ \epsilon_n | \mathcal{F}_n } A  = \frac{C^2}{\norm{ B }_2^2} B^\top B B
\end{align*}
and it is easy to construct examples where the expectation of this is nonzero.

Also in page 14, we find the following inequality
\begin{align*}
(d+2)\left( e^{2(1+\beta)\sum_{k=1}^n \gamma_k}\norm{\theta_0}^2_2+ e^{2(1+\beta)\sum_{k=1}^n \gamma_k}\left(\sum_{k=1}^n \gamma_k e^{-(1+\beta)\sum_{j=1}^{k-1} \gamma_j}\right)^2+\norm{\theta^*}_2^2\right)\norm{\E(\epsilon_n|s_0)}_2\\
\leq (d+2)\frac{\norm{\theta_0}_2^2+1+\norm{\theta^*}_2^2}{(1-\beta)^2}e^{2(1+\beta)\sum_{k=1}^n \gamma_k}\norm{\E(\epsilon_n|s_0)}_2,
\end{align*}
where it is not clear as to how the $(1-\beta)^2$ factor appears in the denominator (also confirmed by a personal communication with one of the authors).


\section{Problems with the Proof: High Probability Bound}
\subsection{Bugs in the paper by \cite{flstda}}
The proof of the high probability bound starts on page 14, in section A.1.
The first problem happens in the display on the bottom of this page in the proof of Lemma 6.
Here, we are told that for $A = \phi \phi^\top - \beta \phi (\phi')^\top$ (we are dropping indices to remove clutter),
\begin{align*}
A^\top A = \norm{\phi}^2 \phi \phi^\top - \beta ( 2 - \norm{\phi}^2\beta ) \phi' (\phi')^\top\,,
\end{align*}
where $\norm{x}$ denotes the $2$-norm of $x$.
However, using $A = \phi( \phi-\beta \phi')^\top$, a direct calculation gives:
\begin{align*}
A^\top A 
& = ( \phi-\beta \phi') \phi^\top \phi ( \phi-\beta \phi')^\top \\
& = \norm{\phi}^2  ( \phi-\beta \phi') ( \phi-\beta \phi')^\top \\
& = \norm{\phi}^2  \left\{ \phi \phi^\top-\beta (\phi' \phi^\top + \phi (\phi')^\top) + \beta^2 \phi' (\phi')^\top \right\}\,,
\end{align*}
which does not match the previous display.
The terms that do not match are the linear-in-$\beta$ terms.
In the first display we have $ -2 \beta \phi' (\phi')^\top$, while in the bottom we have $-\beta (\phi' \phi^\top + \phi (\phi')^\top)$.

The first equality of their display states (in equivalent form) that 
\begin{align*}
A^\top A = \norm{\phi}^2 ( \phi \phi^\top - 2\beta \phi (\phi')^\top +\beta^2 \phi' (\phi')^\top )\,.
\end{align*}
We see that while this is closer to the correct result, here the mistake is that $-2\beta \phi  (\phi')^\top$ is replacing $-\beta(\phi (\phi')^\top + \phi' \phi^\top)$.
\begin{comment}
We also have a hard time following the rest of the proof of this lemma.
In fact, we have a simple proof for a result similar to stated in this lemma.
This works by observing that one needs to 
bound the Lipschitz factor of $\theta \mapsto \bar T_n \dots \bar T_{t+1} ( (I-\gamma_t A_t)\theta + b_t )$, where
$A_t = \phi_t (\phi_t -\beta \phi_t')^\top$, $b_t = r_{i_t} \phi_t$ and
 $\bar T_t(\theta) = (I-\gamma_t \bar A_T) \theta + \bar b_T$ (we assume that the stepsize sequence is deterministic).
 Clearly, this Lipschitz factor can be bounded by $\prod_{s=t+1}^n \norm{I-\gamma_s \bar A_T}_2 \norm{I-\gamma_t A_t}_2$
 (the composition of Lipschitz maps is Lipschitz with a factor that is the product of the individual maps' Lipschitz factors). 
Invoking (A3) then indeed gives some version of Lemma 7.
By making an appropriate assumption (similar to those used in our paper), 
$\prod_{s=t+1}^n \norm{I-\gamma_s \bar A_T}_2 $ can be controlled.
\end{comment}
\subsection{Bugs in the paper by  \cite{lstdicmla}}
(In this paper, $s_t$ is a sequence of states obtained whole following a fixed policy in an MDP.)
In page 10 of \cite{lstdicmla} the expression for $\mathbf{E}[a_{j+1}^\top a_{j+1}]$ contains terms that involve the product of $P$ and $P^\top$. This cannot be correct, as here we can take the expectation first over the next state, which will bring in a \emph{single} instance of $P$.
To remove clutter, drop the $j$ subindex, and set $A = \phi(\phi- \beta \phi')^\top$, where $\phi  = \phi(s_{j})$ and $\phi' = \phi(s_{j+1})$.
The incriminated expression from Eq. (15) of the paper is
\begin{align}
\EE{ A - \tfrac{\gamma}{2} A^\top A  } = \Phi^\top (I - \beta \Psi P - \tfrac{\gamma}{2}(\Delta - \beta P^\top (2I - \beta \Delta)\Psi P) ) \Phi\,.
\label{eq:bad1}
\end{align}
Here, $\Phi$ is the $S\times d$ matrix whose $s$th row is $\phi^\top(s)$ ($s\in \{1, \dots, S\}$),
$\Psi$ is the $S \times S$ diagonal matrix whose $i$th diagonal entry is $\Prob{s_t = i}$,  \todoc{Do they assume stationarity?}
while $\Delta$ is another $S\times S$ diagonal matrix whose $s$th entry is $\norm{\phi(s)}_2^2$.
A direct calculation (as before) gives that
\begin{align}
A^\top A 
& = ( \phi-\beta \phi') \phi^\top \phi ( \phi-\beta \phi')^\top  \nonumber \\
& = \norm{\phi}^2  ( \phi-\beta \phi') ( \phi-\beta \phi')^\top \nonumber \\
& = \norm{\phi}^2  \left\{ \phi \phi^\top-\beta (\phi' \phi^\top + \phi (\phi')^\top) + \beta^2 \phi' (\phi')^\top \right\}\, \nonumber \\
& = \norm{\phi}^2  \phi \phi^\top-\beta \norm{\phi}^2 \phi' \phi^\top -\beta \norm{\phi}^2  \phi (\phi')^\top + \beta^2 
\norm{\phi}^2 \phi' (\phi')^\top\,.
\end{align}
%where we abbreviate the 2-norm with $\norm{\cdot}$.
The expectation of each terms are as follows:
\begin{align*}
\EE{\norm{\phi}^2  \phi \phi^\top } 
& = \sum_s \Prob{s_t = s}\norm{\phi(s)}^2 \phi(s) \phi(s)^\top = \Phi^\top \Delta \Psi \Phi\,,\\
\beta \EE{\norm{\phi}^2 \phi' \phi^\top } 
& = \beta \sum_s \Prob{s_t = s}\norm{\phi(s)}^2 \left\{ \sum_{s'}  P(s'|s) \phi(s')  \right\} \phi(s)^\top 
   = \beta (P \Phi)^\top \Delta \Psi \Phi\,,\\
\beta \EE{\norm{\phi}^2 \phi (\phi')^\top } 
& = \beta \EE{\norm{\phi}^2 \phi' \phi^\top }^\top
= \beta \left\{(P \Phi)^\top \Delta \Psi \Phi\right\}^\top 
=  \beta (\Phi^\top \Delta \Psi P \Phi) \,,\\
\EE{ \beta^2  \norm{\phi}^2 \phi' (\phi')^\top } 
& =\beta^2 \sum_s P(s_t=s) \norm{\phi(s)}^2 \left\{ \sum_{s'} P(s'|s) \phi(s') \phi(s')^\top \right\}\,.
\end{align*}
Further,
\begin{align*}
\EE{ A }  = \Phi^\top  \Psi (I - \beta P) \Phi\,.
\end{align*}
Putting together things we see the mismatch with \eqref{eq:bad1}.
To see this even more clearly, assume that $ \norm{\phi(s)}^2 = 1$ for any $s\in \{1,\dots,S\}$.
Then, $\Delta = I$, and by stationarity, $\Prob{s_{t+1} = s } = \Prob{s_t=s}$, hence,
\begin{align*}
\EE{ \beta^2  \norm{\phi}^2 \phi' (\phi')^\top } & =
\beta^2 \EE{ \phi (\phi)^\top } =\beta^2 \Phi^\top \Psi \Phi\, .
\end{align*}
Thus, 
\begin{align*}
\EE{ A^\top A } & = 
\Phi^\top  \Psi \Phi -  \beta \Phi^\top P^\top  \Psi \Phi - \beta \Phi^\top  \Psi P \Phi + \beta^2 \Phi^\top \Psi \Phi \\
& = 
\Phi^\top  \left(\Psi -  \beta  (P^\top  \Psi +   \Psi P) + \beta^2 \Psi \right) \Phi \\
\end{align*}
and hence
\begin{align*}
\EE{ A - \frac{\gamma}{2} A^\top A } & = 
 \Phi^\top \Psi(I - \beta P) \Phi - \tfrac{\gamma}{2} 
 \Phi^\top  \left(\Psi -  \beta  (P^\top  \Psi +   \Psi P) + \beta^2 \Psi \right) \Phi \\
& = 
\Phi^\top  \left\{ \Psi (I - \beta P)- \tfrac{\gamma}{2} 
\left(\Psi -  \beta  (P^\top  \Psi +   \Psi P) + \beta^2 \Psi \right) \right\} \Phi \,.
\end{align*}
while \eqref{eq:bad1} gives
\begin{align*}
\EE{ A - \tfrac{\gamma}{2} A^\top A  } 
& = \Phi^\top (I - \beta \Psi P - \tfrac{\gamma}{2}(I - \beta P^\top (2I - \beta I)\Psi P) ) \Phi \\
& = \Phi^\top \left\{I - \beta \Psi P - \tfrac{\gamma}{2}(I - 2 \beta P^\top\Psi P  +  \beta^2 P^\top \Psi P) \right\} \Phi\,.
\end{align*}
Choosing $\Phi = I$, we find that the two expressions are equal if and only if
\begin{align*}
 \Psi (I - \beta P)- \tfrac{\gamma}{2} \left( \Psi -  \beta  (P^\top  \Psi +   \Psi P) + \beta^2 \Psi \right)
= 
I - \beta \Psi P - \tfrac{\gamma}{2}(I - 2 \beta P^\top\Psi P  +  \beta^2 P^\top \Psi P)\,,
\end{align*}
which implies, e.g., that $\Psi=I$ (by choosing $\gamma=0$), which is not possible since the diagonal elements of $\Psi$ must sum to one.
Even if we correct the first identity to $\Psi$, we see that we must have
\begin{align*}
 \Psi -  \beta  (P^\top  \Psi +   \Psi P) + \beta^2 \Psi 
 = I - 2 \beta P^\top\Psi P  +  \beta^2 P^\top \Psi P\,,
\end{align*}
which again, means that $\Psi = I$, and also that $P^\top \Psi + \Psi P = P^\top \Psi P$ and that $\Psi = P^\top \Psi P$.
The first equality is always false, and the others are false except (perhaps) in some very special cases.


\section{Issues with the Setup}
\subsection{Boundedness of iterates: \citet{flstda}}
\citet{flstda} assume that the parameter vector stays such that the value function $\Phi \theta$  will be bounded in $L^\infty$-norm (see assumption (A3) of \cite{flstd} and \cite{flstda}). This assumption is critical in establishing Lemma~7 (see pages 15 and 16, \cite{flstda}), in an argument that is similar to the proof of McDiarmid's inequality.
We suspect the following shortcomings with assumption (A3):
\begin{itemize}
\item The assumption is stated in a somewhat sloppy fashion. We take the authors meant to say that $\sup_n \norm{\Phi \theta_n}_\infty<+\infty$ holds almost surely. This seems like a strong assumption:  ensuring this will most likely further restrict the step-size sequences that can be used. The step-size sequences that give the best rate under (A3) may include step-size sequences which in fact lead to $\Prob{\limsup_{n\to\infty}\norm{\theta_n}=\infty}>0$. Without proving that this is not the case, the results of the paper have limited utility.
% We see in our paper that for some cases, the iterates can blow up if the step-size is inappropriately selected!
\item One possibility would be to modify the algorithm by adding a projection step to guarantee boundedness.
It is still unclear  whether this alone would ensure convergence of the error to zero. In any case, the expected error bound analysis is invalidated if a projection step is present (basically, the algebraic identities will all fail to hold) and a new proof will be required.

%For the sake of an argument let us also additionally assume that the iterates are bounded by a projection step (which  \citet{flstda} don't mention explicitly). Then would the recursion unfold the same way as it does in Equation (26) of \cite{flstda} (or Equation (13) of \cite{flstd})? How are we to ensure decay/contraction (as shown in Equation (27) of \cite{flstda}) at each step (since projection might be non-contracting)?
%\item The recursion eventually (see Equation (26) of \cite{flstda}) has only \emph{additive noise} and not \emph{multiplicative noise} as analyzed in our submission. The additive noise is captured in the form of the $H_\beta^2$ term in Equation (28) of \cite{flstd}. This reduction to additive noise is possible due to \textbf{$(A3)$}.
\end{itemize}
\subsection{Boundedness of iterates:  \citet{lstdicmla}}
\citet{gugan} mention that (citing a personal communication with \citet{lstdicmla}) 
that \cite{lstdicmla} assume implicitly a projection step in all the high probability bounds. 
While this implicit projection in itself does not affect the high probability bound proofs directly, 
the algebraic steps are invalidated.
Furthermore, the set that the iterates are projected to should contain the TD(0) solution.
How to ensure this (without knowing $A,b$) remains to be seen.

\subsection{Relation between Covariance Matrix and $\bar{A}_T$ matrix}
\citet{flstda} assume positive definiteness of \textbf{$(A4)$} covariance matrix $\frac{1}{T}\Phi^\top_T \Phi_T$. However, unlike regression problems, in reinforcement learning problems what appears in the recursion (see Equation (6)) is not the covariance matrix, but a different matrix $\bar{A}_T=\frac{1}{T}\sum_{i=1}^T \phi(s_i)(\phi(s_i)-\beta\phi(s_i'))^\top$ defined in pages 2, 4 (below Equation (5)), 8 and 16 of \cite{flstda}. Usually, without a sampling assumption known as the `on-policy' case (see \cite{gtd} for a discussion on `on-policy' vs `off-policy') the eigenvalues of $\bar{A}_T$ cannot be guaranteed to have all positive real parts. While \citet{flstda} mention the `on-policy' sampling in the introduction, there is no explicit sampling assumption in the list of assumption. In fact, we doubt that the proposed algorithm will converge without extra assumption (as discussed above).
\if0
This is worrisome for the following reasons
\begin{itemize}
\item \textbf{$(A4)$} deals only with the covariance matrix and it is not clear how that alone is sufficient to say that $\bar{A}_n-(1-\beta)\mu$ is positive definite (see below Equation (26) page 16 of \cite{flstda}; the problem mentioned earlier).
\item For the sake of the argument, let us suppose that $(1-\beta)\mu$ is the smallest eigenvalue of $\bar{A}_n$. However, the problem here is that $\bar{A}_n$ is not symmetric in general and it is typical in reinforcement learning applications to say a non-symmetric matrix $M$ is positive definite if $x^\top M x>0$ for all real valued vectors $x$. While it follows from this notion of positive definiteness of $M$ that all the eigenvalues of $M$ have positive real parts, there could still be eigenvalues with complex conjugate parts. Hence to assume that the smallest eigenvalue of $\bar{A}_n$ to be $(1-\beta)\mu$ might be problematic (because there could be other complex eigenvalues with smaller magnitude). While it might be true in practice that the smallest eigenvalue of $\bar{A}_n$, is greater than $(1-\beta)\mu$ to our knowledge there is no proof for this fact in literature.
\end{itemize}
\fi
\subsection{Blow Up of the Bound}
We would like to note that the rate expression in Corollary~4 of \cite{flstda} (or Corollary~2 of \cite{flstd}) contains a constant $C$. The authors do mention that the sampling error (a.k.a. variance) blows up as $\alpha \rightarrow 1$.
However, it also looks like that even the constant $C=\sum_{n=1}^\infty \exp(-\mu cn^{1-\alpha})$ (appearing in the bound of the bias) will blow up as $\alpha\rightarrow 1$, in which case it seems that the claim that the $1/\sqrt{n}$ rate can be achieved in the limit will not hold.

\subsection{Doubt about the Effectiveness of the Algorithms}
In Corollary~4 of \cite{flstda} (or Corollary~2 of \cite{flstd}), we learn that the value of $c$ governing the stepsize of the primary update must be in a small range (it must be between $1.33$ and $2$). This means, that effectively, the stepsize $\gamma_n$ behaves as $1/n^\alpha$ ($c$ has very little effect). At least when $\alpha=1$, we know that stepsizes like this make the bias decrease slowly and averaging remains ineffective. This seems to be at odds with the suggestion after this result that $\alpha\to 1$ is a desirable choice. In fact, we would be inclined to choose $\alpha=1/2$, i.e., its lowest value. This is because then the bias is decreased relatively quickly, while the variance will be controlled by the additional averaging. However, given all the problems with this, it remains to be seen whether this is indeed a reasonable choice and under exactly what conditions this is reasonable.
\bibliographystyle{plainnat}
\bibliography{issuesref}
%\newpage
%%!TEX root =  flsa.tex
\appendix
\section{Linear Algebra Preliminaries}\label{sec:appendix}
\subsection{Additional Notations}\label{sec:addnot}
For $x=a+ib\in \C$, we denote its real and imaginary parts by $\re{x}=a$ and $\im{x}=b$ respectively. Given a $x\in \C^d$, for $1\leq i \leq d$, $x(i)$ denotes the $i^{th}$ component of $x$.
For any $x\in \C$ we denote its modulus $\md{x}=\sqrt{\re{x}^2+\im{x}^2}$ and its complex conjugate by $\bar{x}=a-ib$.
We use $A\succeq 0$ to denote that the
square matrix $A$ is Hermitian and positive semidefinite (HPSD):
$A = A^*$, $\inf_x x^* A x\ge 0$. We use $A\succ 0$ to denote that the square matrix $A$ is Hermitian and positive definite (HPD): $A=A^*$, $\inf_x x^* A x > 0$.
For $A,B$ HPD matrices, $A\succeq B$ holds if $A-B\succeq 0$.
We also use $A\succ B$ similarly to denote that $A-B \succ 0$.
We also use $\preceq$ and $\prec$ analogously. We denote the smallest eigen value of a real symmetric positive definite matrix $A$ by $\lambda_{\min}(A)$.\par
We now present some useful results from linear algebra.

Let $B$ be a $\dcd$ block diagonal matrix given by $B=\begin{bmatrix} B_1 &0 &0 &\ldots &0 \\ 0 &B_2 &0 &\ldots &0  \\ \vdots &\vdots &\vdots &\vdots &\vdots \\ 0 &\ldots &0 &0 &B_k \end{bmatrix}$, where $B_i$ is a $d_i \times d_i$ matrix such that $d_i<d,\,\forall i=1,\ldots,k$ (w.l.o.g) and $\sum_{i=1}^k d_i=d$. We also denote $B$ as
\begin{align*}
B=B_1 \op B_2 \op \ldots B_k=\op_{i=1}^k B_i
\end{align*}

\subsection{Results in Matrix Decomposition and Transformation}
We will now recall Jordon decomposition.
\begin{lemma}\label{jordon}
Let $A\in \C^{\dcd}$ and $\{\lambda_i\in \C,i=1,\ldots,k\leq d \}$ denote its $k$ distinct eigenvalues.
There exists a complex matrix $V\in \C^{\dcd}$ such that $A=V\tL V^{-1}$, where
$\tL=\tL_1\op\ldots\op\tL_k$, where each $\tL_i,\,i=1,\ldots,k$ can further be written as $\tL_i= {\tL}^i_{1}\op \ldots \op {\tL}^i_{{l(i)}}$. Each of ${\tL}^i_{j},j=1,\ldots,l(i)$ is a $d^i_j\times d^i_j$ square matrix such that $\sum_{j=1}^{l(i)} d^i_j =d_i$ and has the special form given by
${\tL}^i_{j}=\begin{bmatrix} \lambda_i &1 &0 &\ldots &0 &0\\ 0 &\lambda_i &1 &0 &\ldots &0 \\ 0 &\vdots &\vdots &0 &\lambda_i &1 \\ 0 &\ldots &0 &0 &0 &\lambda_i \end{bmatrix}$.
\end{lemma}

\begin{lemma}\label{lm:simtran}
Let $A\in \C^{\dcd}$ be a Hurwitz matrix. There exists a matrix $U\in \gln$ such that $A=U\Lambda U^{-1}$ and $\Lambda^*+\Lambda$ is a real symmetric positive definite matrix.
\end{lemma}
\begin{proof}
It is trivial to see that for any $\Lambda\in \C^{\dcd}$, $\left(\Lambda^*+\Lambda\right)$ is Hermitian. We will use the decomposition of $A=V \tL V^{-1}$ in \Cref{jordon} and also carry over the notations in \Cref{jordon}. Consider the diagonal matrices $D^i_j=\begin{bmatrix} 1  &0 &0 &\ldots &0 &0\\ 0 &\re{\lambda_i} &0 &0 &\ldots &0 \\ 0 &\vdots &\vdots &0 &\re{\lambda_i}^{d^i_j-1} &0 \\ 0 &\ldots &0 &0 &0 &\re{\lambda_i}^{d^i_j} \end{bmatrix},\,\forall j=1,\ldots,l(i)$, $D^i=D^i_1 \op\ldots\op D^i_{l(i)},\,\forall i=1,\ldots,k$ and $D=D^1 \op\ldots\op D^k$.
It follows that $A=(VD) \Lambda (VD)^{-1}$, where $\Lambda$ is a matrix such that
$\Lambda=\Lambda_1 \op \ldots \op \Lambda_k$, where each $\Lambda_i,\,i=1,\ldots,k$ can further be written as
$A_i=\Lambda^i_{1} \op \ldots \op \Lambda^i_{{l(i)}}$. Each of $\Lambda^i_{j}$ is a $d^i_j\times d^i_j$ square matrix with the special form given by
$\Lambda^i_{j}=\begin{bmatrix} \lambda_i &\re{\lambda_i} &0 &\ldots &0 &0\\ 0 &\lambda_i &\re{\lambda_i} &0 &\ldots &0 \\ 0 &\vdots &\vdots &0 &\lambda_i &\re{\lambda_i} \\ 0 &\ldots &0 &0 &0 &\lambda_i \end{bmatrix}$.

Now we have $\frac{(\Lambda^*+\Lambda)}{2}=\op_{i=1}^k \op_{j=1}^{l(i)}\frac{\Lambda^{i*}_{j}+\Lambda^i_{j}}{2}$, where $\frac{\Lambda^{i*}_{j}+\Lambda^i_{j}}{2}=\begin{bmatrix} \re{\lambda_i} &\frac{\re{\lambda_i}}{2} &0 &\ldots &0 &0\\ \frac{\re{\lambda_i}}{2} &\re{\lambda_i} &\frac{\re{\lambda_i}}{2} &0 &\ldots &0 \\ 0 &\vdots &\vdots &0 &\re{\lambda_i} &\frac{\re{\lambda_i}}{2} \\ 0 &\ldots &0 &0 &\frac{\re{\lambda_i}}{2} &\re{\lambda_i} \end{bmatrix} $. Then for any $x=(x(i),i=1,\ldots,d)\in \C^d (\neq \mathbf{0})$, we have %there exists a $b\in \{-1,1\}^d$, such that
\begin{align*}
x^* \frac{(\Lambda^*+\Lambda)}{2} x &=\re{\lambda_i} \left(\sum_{i=1}^d \bar{x}{(i)} x(i)+\sum_{i=1}^{d-1} \frac{\bar{x}(i) x(i+1) + x(i)\bar{x}(i+1)}{2}\right) \\
&=\frac{\re{\lambda_i}}{2}\left(\md{x(1)}^2+ \md{x(d)}^2\right)+\frac{\re{\lambda_i}}{2}\left( \sum_{i=1}^{d-1} \md{x(i)}^2+\bar{x}(i) x(i+1) + x(i)\bar{x}(i+1)+\md{x(i+1)}^2 \right)\\
&>\frac{\re{\lambda_i}}{2}\left(\sum_{i=1}^d \md{x(i)+x(i+1)}^2 \right)\\
&> 0
\end{align*}
\end{proof}
\section{Proofs}\label{sec:proofs}
\subsection{LSA with CS-PR for Positive Definite Distributions}
%In this subsection, we assume that $P$ satisfies the assumptions in \Cref{assmp:lsa} and in addition $P$ is \emph{positive definite}.
In this subsection, we re-write \eqref{eq:lsa} and \Cref{assmp:lsa} to accomodate complex number computations and in addition assume that $P$ is \emph{positive definite}. To this end,
\begin{subequations}\label{eq:lsacmplx}
\begin{align}
\label{conststepapp}&\text{LSA:} &\theta_t&=\theta_{t-1}+\alpha(b_t-A_t\theta_{t-1}),\\
\label{iteravgapp}&\text{PR-Average:} &\thh_t&=\frac{1}{t+1}{\sum}_{i=0}^{t}\theta_i,
\end{align}
\end{subequations}
where $\thh_t, \theta_t \in \C^{d}$. We now assume,
\begin{assumption}\label{assmp:lsacmplx}
\begin{enumerate}[leftmargin=*, before = \leavevmode\vspace{-\baselineskip}]
\item \label{distapp} $(b_t, A_t)\sim (P^b,P^A), t\geq 0$ is an \iid sequence, where $P^b$ is a distribution over $\C^d$ and $P^A$ is a distribution over $\C^{\dcd}$. We assume that $P$ is positive definite.
\item \label{matvarapp} The martingale difference sequences\footnote{$\EE{M_t|\F_{t-1}}=0$ and $\EE{N_t|\F_{t-1}}=0$} $M_t\eqdef A_t-A_{P}$ and $N_t\eqdef b_t-b_{P}$ associated with $A_t$ and $b_t$ satisfy the following
\begin{align*}\E\left[ \norm{M_t}^2\mid\F_{t-1}\right]\leq \sigma^2_{A_P}, \, \E[N_t^* N_t]=\sigma^2_{b_P}.\end{align*}
\item $A_P$ is invertible and there exists a $\ts=A^{-1}_Pb_P$.
\end{enumerate}
\end{assumption}





We now define the error variables and present the recurison for the error dynamics. In what follows, definitions in \Cref{sec:def} and \Cref{sec:prob} continue to hold.
\begin{definition}\label{def:err}
\begin{itemize}[leftmargin=*, before = \leavevmode\vspace{-\baselineskip}]
\item Define error variables $e_t\eqdef \theta_t-\ts$ and $\eh_t\eqdef \thh_t-\ts$.
\item Define $\forall\, t\geq 0$ random vectors $\zeta_t\eqdef b_t-b-(A_t-A_P)\ts$.
\item Define constants $\sigma_1^2\eqdef\sigma_A^2\norm{\ts}^2+\sigma_b^2$ and $\sigma_2^2\eqdef\sigma_A^2\norm{\ts}$. Note that $\EE{\norm{\zeta_t}^2}\leq \sigma_1^2$ and $\EE{\norm{M_t\zeta_t}}\leq \sigma_2^2$.
%\item Define constants $\EE{\norm{\zeta_t}^2}\eqdef\sigma_1^2$ and $\EE{\norm{M_t\zeta_t}}\eqdef \sigma_2^2$.
\item Define $\forall\,i\geq j$, the random matrices $F_{i,j}=(I-\alpha A_i)\ldots (I-\alpha A_j)$ and $\forall,\,i<j$ $F_{i,j}=\I$.
\end{itemize}
\end{definition}



\paragraph{Error Recursion} Let us now look at the dynamics of the error terms defined by
\begin{align}\label{eq:errrec}
\begin{split}
\theta_t&=\theta_{t-1}+\alpha\big(b_t-A_t\theta_{t-1}\big)\\
\theta_t-\ts&=\theta_{t-1}-\ts+\alpha\big(b_t-A_t(\theta_{t-1}-\ts+\ts)\big)\\
e_t&=(I-\alpha A_t)e_{t-1}+\alpha(b_t -b -(A_t-A)\ts)\\
e_t&=(I-\alpha A_t)e_{t-1}+\alpha\zeta_t
\end{split}
\end{align}

\begin{lemma}\label{lm:pd}
Let $P$ be a distribution over $\C^d\times \C^{\dcd}$ satisfying \Cref{assmp:lsacmplx}, then there exists an $\alpha_P>0$ such that $\rhod{P}>0$ and $\rhos{P}>0,~\forall \alpha \in (0,\alpha_P)$.
\end{lemma}
\begin{proof}
\begin{align*}
\rhos{P}&\stackrel{(a)}{=}\inf_{x:\norm{x}=1}x^* (A_P^*+A_P)x -\alpha x^*\EE{A_t^* A_t} x\\
&\stackrel{(b)}{=}\inf_{x:\norm{x}=1}x^* (A_P^*+A_P)x -\alpha x^* A^*_P A_P -\alpha x^* \EE{M_t^* M_t} x\\
&\stackrel{(c)}{\geq} \lambda_{\min}(A^*_P+A_P)-\alpha \norm{A_P}^2-\sigma^2_A
\end{align*}
The proof is complete by choosing $\alpha_P<\frac{\lambda_{\min}(A^*_P+A_P)}{\norm{A_P}^2+\sigma^2_A}$. Here $(a)$ follows from definition of $\rhos{P}$ in \Cref{def:dist}, $(b)$ follows from the fact that $M_t$ is a martingale difference term (see \Cref{assmp:lsacmplx}) and $(c)$ follows from the fact that for a real symmetric matrix $M$ the smallest eigen value is given by $\lambda_{\min}=\inf_{x:\norm{x}=1} x^* M x $.
\end{proof}





%%%%%%%%%%%%%%%%%%%%%%%%%%%%%%%%%%%%%%%%%%
\begin{lemma}[Product unroll lemma]\label{lem:genunroll}
Let $t>i\ge 1$, $x,y\in \C^d$ be $\F_{i}$-measurable random vectors. Then,
\begin{align*}
\E[x^* F_{t,i+1}y|\F_i]=x^* (I-\alpha A_P)^{t-i} y\,.
\end{align*}
\end{lemma}
%%%%%%%%%%%%%%%%%%%%%%%%%%%%%%%%%%%%%%%%%%
\begin{proof}
By the definition of $F_{t,i+1}$,
and because $F_{t-1,i+1} = (I-\alpha A_{t-1}) \dots (I-\alpha A_{i+1})$ is $\F_{t-1}$-measurable,
as are $x$ and $y$,
\begin{align*}
\EE{x^* F_{t,i+1} y | \F_{t-1} } &= x^\top \EE{ (I-\alpha A_t) | \F_{t-1} } F_{t-1,i+1} y\\
&=x^*  (I-\alpha A_P)  F_{t-1,i+1} y\,.
\end{align*}
By the tower-rule for conditional expectations and our measurability assumptions,
\begin{align*}
\EE{x^* F_{t,i+1} y | \F_{t-2} }
&=x^*  (I-\alpha A_P)  \EE{F_{t-1,i+1} |\F_{t-2}} y\\
&= x^* (I-\alpha A_P)^2 F_{t-2,i+1} y\,.
\end{align*}
Continuing this way we get
\begin{align*}
\EE{x^* F_{t,i+1} y | \F_{t-j} }
= x^* (I-\alpha A_P)^j F_{t-j,i+1} y\,, \quad j=1,2,\dots,t-i\,.
\end{align*}
Specifically, for $j=t-i$ we get
\begin{align*}
\EE{x^* F_{t,i+1} y | \F_{i} }  = x^* (I-\alpha A_P)^{t-i} y\,.
\end{align*}
\end{proof}

%%%%%%%%%%%%%%%%%%%%%%%%%%%%%%%%%%%%%%%%%%
\begin{lemma}\label{noisecancel}
Let $t>i\ge 1$ and let $x\in \C^d$ be a $\F_{i-1}$-measurable random vector. Then,
$\E[x^* F_{t,i+1}\zeta_{i}]=0$.
\end{lemma}
%%%%%%%%%%%%%%%%%%%%%%%%%%%%%%%%%%%%%%%%%%
\begin{proof}
By \Cref{lem:genunroll},
\begin{align*}
\EE{x^* F_{t,i+1} \zeta_i | \F_{i} }  = x^* (I-\alpha A_P)^{t-i} \zeta_i\,.
\end{align*}
Using the tower rule,
\begin{align*}
\EE{x^* F_{t,i+1} \zeta_i | \F_{i-1} }
= x^* (I-\alpha A_P)^{t-i}\EE{ \zeta_i | \F_{i-1} }= 0\,.
\end{align*}
\end{proof}

\begin{lemma}\label{lem:unroll}
For all $t>i\ge 0$, $\E \ip{e_i,F_{t,i+1} e_i}=\E\ip{e_i,(I-\alpha A_P)^{t-i} e_i}$.
\end{lemma}
\begin{proof}
The lemma follows directly from \Cref{lem:genunroll}. Indeed,
$\theta_i$ depends only on $A_1,\dots,A_{i},b_1,\dots,b_{i}$, $\theta_i$ and so is $e_i$ $\F_i$-measurable.
Hence, the lemma is applicable and implies that
\begin{align*}
\EE{ \ip{e_i, F_{t,i+1} e_i} | \F_i } =
\EE{ \ip{e_i, (I-\alpha A_P)^{t-i} e_i} | \F_i }\,.
\end{align*}
Taking expectation of both sides gives the desired result.
\end{proof}

\begin{lemma}\label{innerproduct}
Let $i>j \ge 0$ and let $x\in \R^d$ be an $\F_j$-measurable random vector.
Then,
\begin{align*}
\E\ip{F_{i,j+1}x,F_{i,j+1}x}\leq (1-\alpha \rhos{P})^{i-j}\E\norm{x}^2\,.
\end{align*}
\end{lemma}
\begin{proof}
Note that
$S_t\doteq \EE{ (I-\alpha A_t)^* (I-\alpha A_t) | \F_{t-1} }
= I - \alpha (A_P^* + A_P) + \alpha^2 \EE{ A_t^* A_t | \F_{t-1} }$.
Since $(b_t,A_t)_t$ is an independent sequence, $\EE{ A_t^* A_t|\F_{t-1}} = \EE{ A_1^* A_1 }$.
Now, using the definition of $\rhos{P}$ from \Cref{def:dist}
$\sup_{x:\norm{x}= 1} x^\top S_t x = 1 - \alpha \inf_{x:\norm{x}=1} x^\top (A_P^* + A_P - \alpha \EE{A_1^\top A_1}) x
= 1-\alpha \rhos{P}$.
Hence,
\begin{align*}
&\EE{\ip{F_{i,j+1}x,F_{i,j+1}x}|\F_{i-1} }\\
&= \EE{x^* F_{i-1,j+1}^\top (I-\alpha A_i)^* (I-\alpha A_i) F_{i-1,j+1} x\,|\,\F_{i-1}}\\
&=(x F_{i-1,j+1})^* \, S_i \, F_{i-1,j+1} x\\
&\le (1-\alpha \rhos{P}) \, \ip{ F_{i-1,j+1} x, F_{i-1,j+1} x} \\
& \le (1-\alpha \rhos{P})^2\, \ip{ F_{i-2,j+1} x, F_{i-2,j+1} x} \\
& \quad \vdots \\
& \le (1-\alpha \rhos{P})^{i-j}\, \norm{x}^2\,.
%&= \E[x^\top (I-\alpha H_{j+1})\ldots (I-\alpha H_i)^\top (I-\alpha H_i)\ldots (I-\alpha H_{j+1})x|\F_{i-1}]\\
%&=(1-\alpha\rhos{P}) \E[x^\top (I-\alpha H_{j+1})\ldots (I-\alpha H_{i-1})^\top (I-\alpha H_{i-1})\ldots (I-\alpha H_{j+1})x]
\end{align*}
\end{proof}





\begin{comment}
\begin{lemma}
Let $P$ be a distribution over $\R^d\times \R^{\dcd}$ satisfying \Cref{assmp:lsa}, then there exists an $\alpha_P$ such that $\rhod{P}>0$ and $\rhos{P}>0,~\forall \alpha \in (0,\alpha_P)$.
\end{lemma}
\begin{proof}
Since $P$ is positive definite from \Cref{distpd} it follows that $A_P$ is positive definite. Now
\begin{align*}
\rhos{P}&={\inf}_{x\in \C^d\colon\norm{x}=1}\ip{x,\left((A_P+A_P^*)-\alpha C_P\right)x}\\
&\geq {\inf}_{x\in \C^d\colon\norm{x}=1}\ip{x,\left((A_P+A_P^*)\right)x}-\alpha B^2\\
&\geq \lambda_{\min}(A_P+A_P^*)-\alpha B^2
\end{align*}
By choosing $\alpha_P<\frac{\lambda_{\min}(A_P+A_P^*)}{B^2}$, it follows that $\rhos{P}>0,\forall \alpha \in (0,\alpha_P)$. Further, by noting that $C_P\succeq A_P^* A_P$, it is easy to check that $\rhod{P}>\rhos{P}>0,\,\forall \alpha \in(0,\alpha_P)$.
\end{proof}
\end{comment}



\begin{theorem}\label{th:pdrate}
Let $\eh_t$ be as in \Cref{def:err}. %Let $P$ be a distribution over $\C^d\times\C^{\dcd}$ satisfying \Cref{assmp:lsacmplx}.
Then
\begin{align}
%\EE{\norm{\eh_t}^2}
%\leq
%\left(1+\frac2{\alpha\rhod{P}}\right)\frac{1}{\alpha \rhos{P}} \left(\frac{\norm{e_0}^2}{(t+1)^2}+ \frac{\alpha^2{\sigma}_1^2+ {\sigma}_2^2 (\alpha^2\norm{\ts}+\alpha \norm{e_0})}{t+1} \right)\,.
\E[\norm{\eh_t}^2]
\leq \left(1+\frac2{\alpha\rhod{P}}\right)\, \frac1{\alpha\rhos{P}}\, \,
%\left(\frac{\norm{e_0}^2}{(t+1)^2}+ \frac{\alpha^2(\sigma_P^2\norm{\ts}^2+\sigma_b^2)+\alpha (\sigma_P^2\norm{\ts})\norm{e_0}}{t+1} \right)\,.
\left(\frac{\norm{e_0}^2}{(t+1)^2}+ \frac{\alpha^2(\sigma_1^2)+\alpha \sigma_2^2\norm{e_0}}{t+1} \right)\,.
\end{align}
\end{theorem}
\begin{proof}

\begin{align*}
e_t
& = (I-\alpha A_t) (I-\alpha A_{t-1}) e_{t-2}\\ &+ \alpha (I-\alpha A_t) \zeta_{t-1} +\alpha \zeta_t \\
& \quad \vdots\\
& = (I-\alpha A_t) \cdots (I-\alpha A_1) e_0\\ &+ \alpha (I-\alpha A_t) \cdots (I-\alpha A_2) \zeta_1 \\
& + \alpha (I-\alpha A_t) \cdots (I-\alpha A_3) \zeta_2\\
&  \quad \vdots \\
&+ \alpha \zeta_t\,,
\end{align*}
which can be written compactly as
\begin{align}
\label{eq:etft}
e_t = F_{t,1} e_0 + \alpha (F_{t,2} \zeta_1 + \dots + F_{t,t+1} \zeta_t )\,,
\end{align}
\begin{align*}
\eh_t=\frac{1}{t+1}{\sum}_{i=0}^{t}e_i
=\frac{1}{t+1}&\Big\{{\sum}_{i=0}^{t} F_{i,1} e_0 \\
&+ \alpha \sum_{i=1}^{t} \left(\sum_{k=i}^{t} F_{k,i+1} \right)\zeta_i \Big\} ,
\end{align*}
where in the second sum we flipped the order of sums and swapped the names of the variables that the sum runs over.
It follows that \todoc{We should rather use $C$ instead of $H$ here?}
\begin{align*}
\E[\norm{\eh_t}^2]&=\E\ip{\eh_t,\eh_t}
=\frac{1}{(t+1)^2} \sum_{i,j=0}^t \E\ip{e_i,e_j}\,.
\end{align*}
Hence, we see that it suffices to bound $\EE{\ip{ e_i,  e_j }}$.
There are two cases depending on whether $i=j$. When $i< j$,
\begin{align*}
\E\ip{e_i,e_j}
&=\E \ip{e_i,\big[F_{j,i+1} e_i+\alpha\textstyle\sum_{k=i+1}^j F_{j,k+1}\zeta_{k}\big]}\\
&=\E\ip{e_i,F_{j,i+1} e_i}  \text{(from \Cref{noisecancel})}\\
&=\E\ip{e_i, (I-\alpha A)^{j-i} e_i} \text{(from \Cref{lem:unroll})}
\end{align*}
and therefore
\begin{align*}
\label{inter}
\sum_{i=0}^{t-1}\sum_{j=i+1}^t \E\ip{e_i,e_j}
&=\frac1{\alpha\rhod{P}} {\sum}_{i=0}^{t-1}\E\ip{e_i,e_i}\\
&\leq \frac2{\alpha\rhod{P}}{\sum}_{i=0}^{t}\E\ip{e_i,e_i}\,.
\end{align*}
Since $\sum_{i,j}\cdot{} = \sum_{i=j}\cdot{} + 2 \sum_i \sum_{j>i} \cdot{}$,
\begin{align*}
{\sum}_{i=0}^{t}{\sum}_{j=0}^{t} \E\ip{e_i,e_j}&= \left(1+\frac2{\alpha\rhod{P}}\right){\sum}_{i=0}^{t}\E\ip{e_i,e_i}\,.
\end{align*}
Expanding $e_i$ using \eqref{eq:etft} and then using \Cref{innerproduct} and \Cref{assmp:lsacmplx}
\begin{align*}
\E\ip{e_i,e_i}&=\E\ip{F_{i,1}e_0,F_{i,1}e_0}+\alpha^2{\sum}_{j=1}^i\E\ip{ F_{i,j+1}\zeta_j, F_{i,j+1}\zeta_j}+\alpha\sum_{j=1}^i  \E\ip{F_{i,1} e_0, F_{i,j+1}\zeta_j}\\
&\leq (1-\alpha\rhos{P})^i\norm{e_0}^2+ \alpha^2\frac{{\sigma}_1^2}{\alpha \rhos{P}}+ \alpha \frac{{\sigma}^2_2 \norm{e_0}}{\alpha\rhos{P}}\,,
\end{align*}
and so
\begin{align*}
{\sum}_{i=0}^{t}{\sum}_{j=0}^{t} \E\ip{e_i,e_j}
&\leq \left(1+\frac2{\alpha\rhod{P} }\right)\, \frac1{\alpha\rhos{P}}\, (t(\alpha^2{\sigma}_1^2+\alpha {\sigma}^2_2\norm{e_0}) +\norm{e_0}^2)\,.
\end{align*}
Putting things together,
\begin{align}
\E[\norm{\eh_t}^2]
\leq \left(1+\frac2{\alpha\rhod{P}}\right)\, \frac1{\alpha\rhos{P}}\, \,
%\left(\frac{\norm{e_0}^2}{(t+1)^2}+ \frac{\alpha^2(\sigma_P^2\norm{\ts}^2+\sigma_b^2)+\alpha (\sigma_P^2\norm{\ts})\norm{e_0}}{t+1} \right)\,.
\left(\frac{\norm{e_0}^2}{(t+1)^2}+ \frac{\alpha^2(\sigma_1^2)+\alpha \sigma_2^2\norm{e_0}}{t+1} \right)\,.
\end{align}
\end{proof}

\paragraph{Proof of \Cref{lm:hur}}
\begin{lemma} %\label{lm:hur}
Let $P$ be a distribution over $\R^d\times \R^{\dcd}$ satisfying \Cref{assmp:lsa}, then there exists an $\alpha_{P_U}>0$ and $U\in \gln$ such that $\rhod{P_U}>0$ and $\rhos{P_U}>0,~\forall \alpha \in (0,\alpha_P)$.
\end{lemma}
\begin{proof}
We know that $A_P$ is Hurwitz and from  \Cref{lm:simtran} it follows that there exists an $U\in \gld$ such that  $\Lambda=U^{-1} A_P U$ and $(\Lambda^*+\Lambda)$ is real symmetric and positive definite. Using \Cref{def:simdist}, we have $A_{P_U}=\Lambda$ and from \Cref{lm:pd} we know that there exists an $\alpha_{P_U}$ such that $\rhod{P_U}>0$ and $\rhos{P_U}>0,~\forall \alpha \in (0,\alpha_{P_U})$.

\end{proof}



\begin{lemma}[Change of Basis]\label{lm:cb}
Let $P$ be a distribution over $\R^d\times \R^{\dcd}$ as in \Cref{assmp:lsa} and let $U$ be chosen according to \Cref{lm:hur}. Define $\gamma_t\eqdef U^{-1}\theta_t,\,,\gamma_*\eqdef U^{-1}\ts$, then
\begin{align}
\EE{\norm{\gamma_t-\gamma_*}^2}
\leq
\left(1+\frac2{\alpha\rhod{P_U}}\right)\frac{\norm{U^{-1}}^2}{\alpha \rhos{P_U}}
\left(\frac{\norm{\theta_0-\ts}^2}{(t+1)^2}+ \frac{\alpha^2(\sigma_P^2\norm{\ts}^2+\sigma_b^2)+\alpha (\sigma_P^2\norm{\ts})\norm{\theta_0-\ts}}{t+1} \right)\,.
%\left(\frac{\norm{\gamma_0-\gamma_*}^2}{(t+1)^2}+ \frac{\alpha^2{\sigma}_1^2+ {\sigma}_2^2 (\alpha^2+\alpha \norm{\gamma_0-\gamma_*})}{t+1} \right)\,,
\end{align}
where $\gh_t=\frac{1}{t+1}\sum_{s=0}^t \gamma_s$.
\end{lemma}
\begin{proof}
Consider the modified error recursion in terms of $z_t\eqdef \gamma_t-\gamma_*$
\begin{align}\label{eq:newerrrec}
\begin{split}
e_t&=(I-\alpha A_t)e_{t-1}+\alpha\zeta_t\\
U^{-1}e_t&=(I-\alpha U^{-1}A_t U) U^{-1}e_{t-1}+ \alpha U^{-1}\zeta_t\\
z_t&=(I-\alpha \Lambda_t) z_{t-1}+\alpha H_t,
\end{split}
\end{align}
where  $\Lambda_t=U^{-1}A_t U$ and $H_t=U^{-1}\zeta_t$. Note that the error recursion in $z_t$ might involve complex computations (depending on whether $U$ has complex entries or not), and hence \eqref{eq:lsacmplx} and \Cref{assmp:lsacmplx} are useful in analyzing $z_t$.
We know that $\EE{\norm{H_t}^2}\leq \norm{U^{-1}}^2\EE{\norm{\zeta_t}}$ and $\EE{\norm{\Lambda_t H_t}}=\EE{\norm{U^{-1}A_t UU^{-1}\zeta_t}}=\EE{\norm{U^{-1}A_t \zeta_t}}\leq \norm{U^{-1}}\EE{\norm{A_t\zeta_t}}=\norm{U^{-1}}\sigma_2^2$. Now applying \Cref{th:pdrate} to $\hat{z}_t\eqdef \frac{1}{t+1}\sum_{s=0}^t z_t$, we have
\begin{align}
\E[\norm{\zh_t}^2]
&\leq \left(1+\frac2{\alpha\rhod{P_U}}\right)\, \frac1{\alpha\rhos{P_U}}\, \,
\left(\frac{\norm{z_0}^2}{(t+1)^2}+ \frac{\alpha^2(\norm{U^{-1}}^2\sigma_1^2)+\alpha (\norm{U^{-1}}\sigma_2^2)\norm{z_0}}{t+1} \right)\,\\
&\leq \left(1+\frac2{\alpha\rhod{P_U}}\right)\, \frac1{\alpha\rhos{P_U}}\, \,
\left(\frac{\norm{U^{-1}}^2\norm{e_0}^2}{(t+1)^2}+ \frac{\alpha^2(\norm{U^{-1}}^2\sigma_1^2)+\alpha (\norm{U^{-1}}\sigma_2^2)\norm{U^{-1}}\norm{e_0}}{t+1} \right)\,
\end{align}

\end{proof}

\paragraph{Proof of \Cref{th:rate}}
Follows by substituting $\theta_t=U\gamma_t$ in \Cref{lm:cb}.
\begin{comment}
\begin{theorem}[Lower Bound]
There exists a distribution $P$ over $\R^d\times \R^{\dcd}$ satisfying \Cref{assmp:lsa}, such that $\alpha_P>0$, such that $\rhos{P}>0$ and $\rhod{P}>0,\,\forall \alpha\in (0,\alpha_P)$ and
\begin{align*}
\EE{\norm{\thh_t-\ts}^2} &\geq \frac{1}{(t+1)^2}(\alpha)^{-2}(\rhod{P}\rhos{P})^{-1}\Big( \big(1-(1-\alpha \rhos{P})^t\big) \norm{\theta_0-\ts}^2 \\&+ \sum_{s=1}^t \big(1-(1-\alpha \rhos{P})^{(t-s)}\big) \big(\alpha^2(\sigma_A^2\norm{\ts}^2+\sigma_b^2)+\alpha (\sigma_A^2\norm{\ts})\norm{\theta_0-\ts}\big) \Big)
\end{align*}
\end{theorem}
\end{comment}
\paragraph{Proof of \Cref{th:lb}}

Consider the LSA with $(b_t,A_t)\sim P$ such that $b_t=(N_t,0)^\top\in\R^2$ is a zero mean \iid random variable with variance $\sigma^2_b$, and $A_t=A,\,\forall t\geq 0$, where $A=A_P=\begin{bmatrix} \lambda_{\min} &0\\ 0& \lambda_{\max}\end{bmatrix}$, for some $\lambda_{\max}>\lambda_{\min}>0$. Note that in this example $\ts=0$.
By choosing $\alpha<\frac2{\lambda_{\max}}$, in this case it is straightforward to write the expression for $\eh_t$ explicitly as below:
\begin{align*}
\eh_t&=\frac{1}{t+1}\sum_{s=0}^t e_t = \frac{1}{t+1}\sum_{s=0}^t (I-\alpha A_P)^{t-s} e_0 + \sum_{s=1}^t \sum_{i=s}^t (I-\alpha A_P)^{i-s} b_s\\
&=\frac{1}{t+1}(\alpha A_P)^{-1}\left[\left(I-(I-\alpha A_P)^{t+1}\right)e_0 + \sum_{s=1}^t \left(I-(I-\alpha A_P)^{t+1-s}\right) b_s\right]\,.
\end{align*}
Thus,
\begin{align*}
\EE{\norm{\eh_t}^2}&\stackrel{(a)}{=}\frac{1}{(t+1)^2}\Big[\norm{(\alpha A_P)^{-1}\left(I-(I-\alpha A_P)^{t+1}\right)e_0}^2 \\ 
&+\sum_{s=1}^t \norm{(\alpha A_P)^{-1}\left(I-(I-\alpha A_P)^{t+1-s}\right)b_s}^2\Big]\,,
%&\leq\frac{1}{(t+1)^2}\alpha^2\norm{A_P^{-1}}^2\norm{e_0}^2
\end{align*}
and hence
\begin{align*}
\EE{\norm{\eh_t}^2}
& \geq \EE{\eh^2_t(1)}\stackrel{(b)}{=}\frac{1}{(t+1)^2}(\alpha \lambda_{\min})^{-2}\Big[\left(1-(1-\alpha \lambda_{\min})^{t+1}\right)^2 \theta^2_0(1)\\
& + \frac{1}{(t+1)^2}\sum_{s=1}^t\left(1-(1-\alpha \lambda_{\min})^{t+1-s}\right)^2 b^2_s(1) \Big]\,.
\end{align*}
Here $(a)$ and $(b)$ follows from the \iid assumption. Note that in this example, $\rhos{P}=\rhod{P}=2\lambda_{\min} -\alpha \lambda_{\min}^2=\lambda_{\min}(2-\alpha \lambda_{\min})$, and $\norm{\ts}=0$ and $\sigma^2_A=0$. Further, the result follows by noting the fact that noting the fact that $\norm{b_t}^2=b_t(1)^2$ and $\norm{\theta_t}^2=\theta_t(1)^2$.
%\begin{align}
%\EE{\norm{\eh_t}^2}\geq \frac{1}{(t+1)^2}(\alpha \lambda_{\min})^{-2}(1-\alpha \lambda_{\min}^t) \theta^2_0(1)
%\end{align}
\begin{comment}
\textbf{Proof of }
Given any $A_P$ we can choose a unitary matrix $U$ (i.e., $U^\top U=I$) such that $U^\top A_P U=D_P$, where $D_P$ is a diagonal matrix. Thus with $\gamma=U\top \theta$, we have
\begin{align}
\theta_{t}=(I-\alpha A_P)\theta_{t-1}+\alpha b_t\\
\label{eq:gamrec} \gamma_t=(I-\alpha D_P)\gamma_{t-1}+\alpha \zeta_t,
\end{align}
where $\zeta_t=U^\top b_t$. Note that the unitary transformation preserves norm, i.e., $\norm{\gamma}=\theta^\top UU^\top \theta=\norm{\theta}$, $\norm{\theta}_A=\norm{\gamma}_{\Lambda}$ and $\EE{\zeta_t\zeta_t^\top}=\EE{U^\top b_t b_t^\top U}\leq D_P$. Since, $\Lambda_P$ is a diagonal, \eqref{eq:gamrec} has $d$ separate $1$-dimensional equations. In what follows, without loss of generality, all quantities are in $1$-dimension. Also, $\ts=\gamma_*=\mathbf{0}$. Also, let $e_0=\gamma_0$
\begin{align}
\eh_t=\frac{1}{t+1}(\alpha \Lambda)^{-1}\big([I-(I-\Lambda D)^{t+1}]e_0
+\alpha\sum_{s=1}^t [I-(I-\alpha \Lambda)^{t+1-s}] \zeta_s]\big)
\end{align}
Now for small $t$ such that $\alpha \Lambda t<I$,  we have $I- (I-\alpha \Lambda)^t\approx \alpha \Lambda t$. Thus, we have 
\begin{align}
\eh_t \approx \frac{1}{t+1}\big((t+1)e_0
+\alpha\sum_{s=1}^t (t+1-s) \zeta_s]\big)
\end{align}
Taking expectation, we have
\begin{align}
\EE{\norm{\eh_t}}&\approx \frac{1}{(t+1)^2} \big(\B (t+1)^2+\alpha^2 \sum_{s=1}^{t} s^2 \sigma^2_{b_P}),
%\EE{\norm{\eh_t}_\Lambda}&\approx \frac{1}{(t+1)^2} \big( (t+1)e_0^\top \Lambda e_0 +\sum_{s=1}^{t} s^2 \zeta_t^\top \Lambda \zeta_t)\\
\end{align}
The $\sum_{s=1}^t s^2=O(t^3)$, and hence we have $\EE{\norm{\eh_t}}\approx \frac{1}{(t+1)^2} \big(\B (t+1)^2+\alpha^2 O(t^3) \sigma^2_{b_P})$. 
Similarly, we have

$\EE{\norm{\eh_t}_\Lambda}\approx \frac{1}{(t+1)^2} \big((t+1)^2 \B \Lambda  +\alpha^2 O(t^3) \sigma^2_{b_P})$. Now, using $\alpha \Lambda t <1$, it follows that $\EE{\norm{\eh_t}_\Lambda}\approx \frac{1}{(t+1)^2} \big((t+1)^2\B \Lambda  +\alpha^2 O(t^3) \sigma^2_{b_P} \Lambda)\approx O(\frac{1}{\alpha (t+1)}+\alpha \sigma^2_{b_P})$. Further, in the case of $\P_{SPDSN}$, it follows $\sigma^2_{b_P}<\Lambda$, we have $\EE{\norm{\eh_t}_\Lambda}\approx \frac{1}{(t+1)^2} \big((t+1)^2\B \Lambda  +\alpha^2 O(t^3) \sigma^2_{b_P} \Lambda)\approx O(\frac{1}{\alpha (t+1)}+\frac{1}{t})$.

\end{comment}

For any $P$ in $\P_{SPD}$ or $\P_{SPDSN}$, the corresponding $A_P$ matrix is real symmetric and positive definite. Thus, for each problem instance $\exists U$ such that $U^\top A_P U=\Lambda_P$. Define $\zeta_t\eqdef U^\top b_t$, and $\gamma=U^\top \theta$.




It is trivial to check that any $\alpha \in (0,1)$ is a universal step-size for $\P_{USN}$ and $\P_{SN}$. Now for a given $t$, pick $A_P\in (0,1)$ such that $\alpha A_P t=1$. Also, note that $\ts=0$ for any problem in $\P_{USN}$ and $\P_{SN}$. Now, let $\eh_t\eqdef \thh_t-\ts$, we have

\begin{align}
\eh_t=\frac{1}{t+1}(\alpha A_P)^{-1}\big([I-(I-\alpha A_P)^{t+1}]e_0
+\alpha\sum_{s=1}^t [I-(I-\alpha A_P)^{t+1-s}] b_s]\big)
\end{align}
 We have for $t\geq 0$ such that $\alpha A_P t\leq 1$,  $I- (I-\alpha A_P)^t\approx(\alpha A_P t)$. Thus, we have 
\begin{align}
\eh_t \approx\left(\frac{1}{t+1}[ (t+1)e_0
+\alpha\sum_{s=1}^t (t+1-s)b_s]\right)
\end{align}
Taking expectation, we have with $\B=\norm{e_0}^2$
\begin{align}
\EE{\norm{\eh_t}}\approx\left( \frac{1}{(t+1)^2} [\B (t+1)^2+\alpha^2 \sum_{s=1}^{t} s^2 \sigma^2_{b_P}]\right),
\end{align}
The $\sum_{s=1}^t s^2= O(t^3)$, and hence we have $\EE{\norm{\eh_t}}\approx\left(\frac{1}{(t+1)^2} [\B (t+1)^2+\alpha^2 O(t^3) \sigma^2_{b_P}]\right)$. Similarly, we have 
\begin{align}
\EE{\norm{\eh_t}_{A_P}}\approx\left( \frac{1}{(t+1)^2} [(t+1)^2 \B A_P  +\alpha^2 O(t^3) \sigma^2_{b_P}]\right)
\end{align}
Now, using $\alpha A_P t <1$, it follows that 
\begin{align}
\EE{\norm{\eh_t}_{A_P}}& \approx\left(\frac{1}{(t+1)^2} [(t+1)^2\B A_P  +\alpha^2 O(t^3) \sigma^2_{b_P} A_P]\right) \\
&\approx\left(\frac{1}{\alpha (t+1)}+\alpha \sigma^2_{b_P}\right)
\end{align}
Further, in the case of $\P_{SPDSN}$, it follows $\sigma^2_{b_P}<A_P$, we have 
\begin{align}
\EE{\norm{\eh_t}_{A_P}}&=\approx \left(\frac{1}{(t+1)^2} [(t+1)^2\B \Lambda  +\alpha^2 O(t^3) \sigma^2_{b_P} A_P] \right)\\ 
&\approx\left(\frac{1}{\alpha (t+1)}+\frac{1}{t}\right)
\end{align}



\textbf{Proof of \Cref{prop:unistep}}
\begin{proof}
The eigenvalues of $I-\alpha A_P$ are $1-\alpha u\pm i v$, and we can ensure that their modulus is less than unity only when $\alpha<\frac{u}{\sqrt{u^2+v^2}}$. By, letting $u\ra 0$ we can make $\alpha\ra 0$.
\end{proof}

\begin{lemma}\label{lm:matnorm}
Let $A$ be a $\dcd$ matrix with $B\eqdef\max_{ij}\left|A_{ij}\right|$. It follows that $\max_{x\in \R^d: \norm{x}\leq1}x^\top A^\top A x <B^2d$.
\end{lemma}

We now resort to an the following notation for real symmetric positive definite $\dcd$ matrices $C$ and $D$: $C\succ D$ if $C-D$ is positive definite.
\begin{lemma}\label{lm:schur}
Let $A,B,C$ $\dcd$ real matrices. Given a symmetric matrix $M=\left[\begin{matrix}A&B \\B^\top &C\end{matrix}\right]$ be a given $2d\times 2d$ matrix, it follows that $M\succ 0$ if	$A\succ$ and $C-B^\top A^{-1}B\succ 0$.
\end{lemma}

\begin{lemma}\label{lm:amat}
Let $\delta\in(0,1)$ a given discount factor and let $\pi$ be a policy. It follows that the matrix $A_\delta=\Phi^\top D_\pi(I-\delta P_\pi)\Phi$ is positive definite ($x^\top Ax>0,\forall x \in \R^d$) $\forall \delta\in(0,1)$.
\end{lemma}

\textbf{Proof of \Cref{th:tdadmis}}
\begin{proof}
In what follows, we need to show for the said values of $\alpha$, $\rhos{P}>0$ which is equivalent to showing $\max_{x\in \R^d: \norm{x}\leq 1}\E{x^\top(I-\alpha A_t)^\top (I-\alpha A_t)x}<1$. We show it for the two cases
Case $1:$  TD(0) with SFOS. In what follows we use the fact that for $\alpha=\frac{1}{B^2d \rho_{\max}}\phi^\top_t\phi_t<\beta$ for some $\beta\in (0,1)$, and that $x^\top Ax>0,\forall x \in \R^{d}$ We have
\begin{align*}
x^\top(I-\alpha A_t)^\top (I-\alpha A_t)x=&x^\top\big( I- 2\phi_t(\phi_t-\gamma\phi'_t)^\top +\beta\phi_t(\phi-\gamma\phi'_t)^\top-\beta\gamma\phi'_t(\phi_t-\gamma{\phi'}_t)^\top \big)x\\
<&x^\top\big( I- \phi_t(\phi_t-\gamma\phi'_t)^\top -\beta\gamma\phi'_t\phi^\top_t+\beta\gamma^2\phi'_t{\phi'}_t^\top \big)x\\
=&x^\top\big( I- \phi_t(\phi_t-\gamma\phi'_t)^\top -\beta\gamma^2(\gamma\phi'_t\phi^\top_t- \phi'_t{\phi'}_t^\top)-\beta\gamma(1-\gamma^2)\phi'_t\phi_t^\top \big)x\\
\end{align*}
Taking $\E$, we have
\begin{align*}
&\E{x^\top(I-\alpha A_t)^\top (I-\alpha A_t)x}^\top\\
 <&x^\top\big( I- A -\beta\gamma^2(-A)-\beta\gamma(1-\gamma^2)\E{\phi'_t\phi_t^\top} \big)x\\
<&x^\top\big( I- (1-\gamma^2)[A +\beta\gamma\E{\phi'_t\phi_t^\top}] \big)x
\end{align*}
Now we have 
\begin{align*}
x^\top\big(A +\beta\gamma\E{\phi'_t\phi_t^\top}\big)x=& x^\top\big(\E{\phi_t\phi_t}-\gamma\E{\phi_t\phi'_t} +\beta\gamma\E{\phi'_t\phi_t^\top}\big)x\\
=&x^\top\big(\E{\phi_t\phi_t}-\delta\E{\phi_t\phi'_t} )x,
\end{align*}
where $\delta=\gamma(1-\beta)$. However it follows that $x^\top\big(\E{\phi_t\phi_t}-\delta\E{\phi_t\phi'_t} )x>0$ from \Cref{lm:amat}.

Case $2:$ The GTD updates in \Cref{tb:tdalgos} can be expressed as $x_{t+1}=x_t+\alpha (g_t -H_t x_t)$, where $x_t=\left[\begin{matrix}y_t\\\theta_t\end{matrix}\right]$, $H_t=\left[\begin{matrix}I &A_t \\*(1-\alpha)A^\top_t & \alpha A_t^\top A_t\end{matrix}\right]$, and $g_t=\left[\begin{matrix} b_t\\ A_t^\top b_t\end{matrix}\right]$, where $A_t=\phi_t(\phi_t-\gamma{\phi'}_t)^\top$. To show that $\alpha=\frac{1}{2B^4 d^2 \rho_{\max}}$ is a valid witness, we need to show that  $\max_{x\in \R^d}\E{x^\top(I-\alpha H_t)^\top (I-\alpha H_t)x}<1$, which is equivalent to showing $\min_{x\in \R^d:\norm{x}=1}\E{x^\top[(H_t+H_t^\top) -\alpha H_t^\top H_t]x}>0$.

Now $H_t+H_t^\top=\left[\begin{matrix} 2I & \alpha A_t\\ \alpha A_t^\top 2\alpha A_t^\top A_t\end{matrix}\right]$ and $H_t^\top H_t=\left[\begin{matrix} I+(1-\alpha)^2A_tA_t^\top & A_t-\alpha(1-\alpha)A_tA_t^\top A_t\\ A_t^\top-\alpha(1-\alpha)A_t^\top A_t A_t^\top & A_t^\top A_t+\alpha^2A_t^\top A_t A_t^\top A_t\end{matrix}\right]$.  We see that $(H_t+H_t^\top) -\alpha(H_t^\top H_t) \succ M_t$, where $M_t=\left[\begin{matrix}I &\alpha^2(1-\alpha) A_tA_t^\top A_t\\ \alpha^2(1-\alpha) A_t^\top A_t A_t^\top &\alpha A_t^\top A_t-\alpha^3A_t^\top A_tA_t^\top A_t\end{matrix}\right]$. Now from \Cref{lm:schur}, to show that $M_t\succ 0$ we need to ensure 
\begin{align}
&\alpha A_t^\top A_t -\alpha^3A_t^\top A_t A_t^\top A_t - \alpha^4(1-\alpha)^2A_t^\top A_t A_t^\top A_t A_t^\top A_t\\
=&\alpha A_t^\top\big(I-\alpha^2 A_tA_t^\top-\alpha^3A_tA_t^\top A_tA_t^\top\big)A_t\\
\succ &\alpha A_t^\top\big(I-\alpha^2 A_t(I+\alpha A_t^\top A_t)A_t^\top\big)A_t\\
\succ& \frac{\alpha}{2} I,
\end{align}
where the last inequality follows from the choice of $\alpha$, bound $B$ on the features and \Cref{lm:matnorm}
\end{proof}
%\left[\begin{matrix}\end{matrix}\right]




%$f_m(\theta_{m-1})=(\theta^\top \phi(s_{i_m})-(r_{i_m}+\beta\theta^\top \phi(s_{i_{m+1}})))$
\end{document}
